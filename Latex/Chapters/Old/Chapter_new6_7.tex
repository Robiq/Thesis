% Chapter 7
%
\chapter{Comparison and Security evaluation} % Main chapter title
%
\label{Chapter6} % For referencing the chapter elsewhere, use \ref{Chapter7}
%
% TODO Review
To understand the contribution and usability of SendIt, an analysis of existing solutions is needed. By comparing SendIt with existing solutions it is possible to get an idea of what has been improved, what stayed the same, and what still needs to be worked on. In the following section the e-mail system, a generic SNS system and SendIt will all be evaluated and analyzed.
%
\section{Evaluation of SendIt}
%
	SendIt has few weaknesses compared to the other two systems. The main advantage is the direct transfer, which greatly mnimizes the attack surface compared to the other two solutions. This evaluation will not include all attacks made possible by the technology chosen, but rather focus on the weaknesses that come as a consequence of the proposed system. By this it is meant that attacks that are present in all JavaScript solutions will not be discussed. If the attack is possible because of the way JavaScript is implemented in this solution, it will be included.

	%First trust!!!
	The main weakness, as mentioned previously, is the first trust. This can allow an attacker to appear as the legitimate owner of an identity, without this being the case. This is a risk that any user should be aware of, as it is a critical part of the system. Even though it is a part of the design choice, it has to be acknowledged that this is less than optimal.

	The mitigation for such attacks is split in two parts. The first is relying on the user to evaluate whether the user is legitimate. By using social cues, like timing, each user should evaluate if it is likely that the other identity would initate a transfer. One can also contact the other endpoint and confirm that it is actually them. This is of cource only necessary for the first connection.

	The second is the trust model, which is an automatic part of the system. It combats this issue by sharing information and allowing users to have as much data as possible, on all identities. This allows for detection of false identities and will eventually make these identities untrusted entities in the whole system.

	%ACS!!!
	Another attack vector is the ACS service. If it is not trusted, it can pose a security risk, as the endpoint has to rely on the fact that the ACS server relays the data to the correct recipient. If the data sent is encrypted, it will not pose any immediate risk, but it will stop the endpoint from connecting to anyone, as the other endpoint will not get authenticated properly. If it is the first time these endpoints are connecting, however, it puts the victim at risk of being put in touch with the wrong identity.

	The ACS server can also act as a logger that keeps track of which identities are communicating with each other. While not directly an attack, it can raise privacy issues. Since it allows for monitoring of traffic, it can also be argued that it can be used for reconnosance, which allows for a targeted attck on one of the endpoints. Because of this, endpoints should only use ACS servers they trust.

	%MITM
	The ACS server can also deploy MitM attacks, during the connection setup, which leads to the attacker gaining full access to the data transferred. The biggest problem with this type of attack, is that in most cases it cannot be detected until after the data has already been shared and the attacker has full access to it. It will eventually be detected by the trust system, but at that point it might already be too late. While there is also a possiblity for this to happen during the connection setup of the Serverless mode, it is much harder to do, since there is no pre-determined channel for sharing the offer, and as such the attacker has no set point to attack. 

	%XSS
	Finally, there is a theoretical possibility of having XSS attacks executed in the client, since the Serverless mode relies on user input. While there are countermeasures deployed, one can never guarantee that there are no such security holes.
%
\section{E-mail system comparison}
	%E-mail
	%Anything is an improvement
	Comparing the e-mail system to SendIt is a little like comparing your mailbox to a safe. While the mailbox can be locked and it can be made more secure, it is still outside your house, with no access control, not made to withstand attacks, and ultimately, it can just be stolen as a whole. In contrast the safe is made to withstand attacks, and store your data privately.

	In a similar fashion, e-mail was made to send and receive information, like a mailbox. SendIt is made to make this process secure against outside tampering and allow for minimal risk while exchanging information. To continue the analogy, it is like meeting the Sender with your safe, giving him a small slot to put the mail in, that only fits the letter you expect to receive, and immediately returning back to the safety of your house. To summarize, the email system offers some secure solutions, but guarantees none. In practice that means one should assume that none is applied. As such, anything is an improvement.

	%No temporary storage - direct transfer
	The first major advantage SendIt has over traditional e-mails is that it transfers the file directly. None of the issues stemming from files being stored in transit, applies to SendIt. These issues range from attackers hacking the server and getting access to the files, to allowing attackers to monitor data going through the server, since it will always be routed through. As a general rule, a resource is more secure, the less copies exists, since it limits the attack surface.

	%Files will be broken if given enough time
	Another point to take into consideration is that even if only encrypted data is stored on a server, if an attacker gets access to the data, it is only a question of how long it takes to break the encryption. As such, minimizing the risk of leaking any data should be considered critical. This is why direct transfer of files offer such a big advantages.

	%Lower attack surface - direct transfer (TIME!)
	The attack surface which is exposed is also important to evaluate. Having a server store the files, means the attack surface is quite big, since it is always online. With SendIt's direct transfer function, the file is only available during transfer. The time difference for how long the file is available to be attacked is huge. As such, this is another advantage of direct transfer, since the attack surface is significantly reduced.

	%No possibility to impersonate real Sender
	Sender and Receiver guarantee is another big advantage of SendIt. In the e-mail system the Senders identity can be forged easily, which is not the case for SendIt. It also has some issues where the data sent is not kept confidential and can easily be stolen by a passive listening attack. These two combined makes phishing attacks as well as targeted exploits easier to create, since you know what data or information the endpoint is seeking or expecting. By sending this data from a forged address, the chances of successful attacks increase.

	%MALl content received
	In the ordinary e-mail system, one is also not able to decide which content to receive or not. If someone sends an e-mail attachment, it will be delivered to the receiver's PC, or at least their e-mail server. This is unfortunate, since receiving malicious files may cause them to be executed at a later time, or by a unintended action. It is also possible that they may execute without the user knowingly doing so. As such, it is optimal for the user to be able to stop the transfer of unwanted files, like SendIt supports.

	%Automated spreading
	Automatic spreading of malicious files is a big problem today. It is spread to all the victims contacts and keeps spreading in a similar fashion. While SendIt has no direct solution to mitigate this, the fact that it does not support multicast, and that the files cannot be automatically received, will naturally take care of resolving this issue.

	% - Needs connection setup
	The disadvantage of SendIt is that it is required to complete a connection setup phase in order to communicate with the other endpoint. The e-mail system has no such need and one can easily transfer information at any time without the user having to worry about the connection setup.

	% - Synchronous
	Another case where SendIt has different behavior, is during the actual transfer. While the advantages of direct transfers have been clearly indicated, they do lack the ability to support asynchronous transfer. This is a point where the regular e-mail solution has the edge over SendIt. It can be argued, however, that it is a small obstacle to overcome, for such an increase in security.

	% - First trust
	Finally, the e-mail system does not have to rely on the trusted first setup, that SendIt does. 

	%
\section{SNS system comparison}
	%SNS

	%No temporary storage - direct transfer

	%Files will be broken if given enough time

	%Lower attack surface - direct transfer (TIME!)

	%'E2E encryption' not real - skype!

	%Hard to switch identities

	% - Needs connection setup

	% - First trust